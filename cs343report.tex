%%%%%%%%%%%%%%%%%%%%%%%%%%%%%%%%%%%%%%%%%
% Thin Sectioned Essay
% LaTeX Template
% Version 1.0 (3/8/13)
%
% This template has been downloaded from:
% http://www.LaTeXTemplates.com
%
% Original Author:
% Nicolas Diaz (nsdiaz@uc.cl) with extensive modifications by:
% Vel (vel@latextemplates.com)
%
% License:
% CC BY-NC-SA 3.0 (http://creativecommons.org/licenses/by-nc-sa/3.0/)
%
%%%%%%%%%%%%%%%%%%%%%%%%%%%%%%%%%%%%%%%%%

%----------------------------------------------------------------------------------------
%	PACKAGES AND OTHER DOCUMENT CONFIGURATIONS
%----------------------------------------------------------------------------------------

\documentclass[a4paper, 12pt]{article} % Font size (can be 10pt, 11pt or 12pt) and paper size (remove a4paper for US letter paper)
\usepackage{inconsolata}

\usepackage{color}

\definecolor{pblue}{rgb}{0.13,0.13,1}
\definecolor{pgreen}{rgb}{0,0.5,0}
\definecolor{pred}{rgb}{0.9,0,0}
\definecolor{pgrey}{rgb}{0.46,0.45,0.48}

\usepackage[protrusion=true,expansion=true]{microtype} % Better typography
\usepackage{graphicx} % Required for including pictures
\usepackage{wrapfig} % Allows in-line images
\usepackage{listings}
\lstset{language=Java,
  showspaces=false,
  showtabs=false,
  breaklines=true,
  showstringspaces=false,
  breakatwhitespace=true,
  commentstyle=\color{pgreen},
  keywordstyle=\color{pblue},
  stringstyle=\color{pred},
  basicstyle=\ttfamily,
  moredelim=[il][\textcolor{pgrey}]{$$},
  moredelim=[is][\textcolor{pgrey}]{\%\%}{\%\%}
}

\usepackage{mathpazo} % Use the Palatino font
\usepackage[T1]{fontenc} % Required for accented characters
\linespread{1.25} % Change line spacing here, Palatino benefits from a slight increase by default
%\usepackage[1.0in]{geometry}
\makeatletter
\renewcommand\@biblabel[1]{\textbf{#1.}} % Change the square brackets for each bibliography item from '[1]' to '1.'
\renewcommand{\@listI}{\itemsep=0pt} % Reduce the space between items in the itemize and enumerate environments and the bibliography

\renewcommand{\maketitle}{ % Customize the title - do not edit title and author name here, see the TITLE block below
\begin{flushright} % Right align
{\LARGE\@title} % Increase the font size of the title

\vspace{30pt} % Some vertical space between the title and author name

{\large\@author} % Author name
\\\@date % Date

\vspace{30pt} % Some vertical space between the author block and abstract
\end{flushright}
}

%----------------------------------------------------------------------------------------
%	TITLE
%----------------------------------------------------------------------------------------

\title{\textbf{Exploring DNS Security}\\ % Title
Focused and Deliciously Witty Subtitle} % Subtitle

\author{\textsc{J. Berretta, J. Capuder, E. Guendel} % Author
\\{\textit{Oberlin College}}} % Institution

\date{\today} % Date

%----------------------------------------------------------------------------------------

\begin{document}

\maketitle % Print the title section

%----------------------------------------------------------------------------------------
%	ABSTRACT AND KEYWORDS
%----------------------------------------------------------------------------------------

%\renewcommand{\abstractname}{Summary} % Uncomment to change the name of the abstract to something else

\begin{abstract}
The security of common internet protocols is an important issue when considering how to satisfy the requirements of security and privacy as technology becomes more integral to our daily lives. The components required to simulate a DNS lookup were implemented in Java. This simulation was used to attempt a common DNS cache poisoning attack, and explore methods for improving the security of DNS protocol.
\end{abstract}

%\hspace*{3,6mm}\textit{Keywords:} lorem , ipsum , dolor , sit amet , lectus % Keywords

\vspace{10pt} % Some vertical space between the abstract and first section

%----------------------------------------------------------------------------------------
%	ESSAY BODY
%----------------------------------------------------------------------------------------

\section*{Introduction}

One of the most important requirements of any communication network, such as the internet or the United States Postal Service, is the ability to uniquely identify and locate participants of the network. This requirement is often met by assigning some form of address to participants: to determine the location to which an outgoing letter should be delivered, the sender must specify country of the recipient, what part of the country they live in, and finally the specific street/number combination at which the recipient resides. This system, however, was designed for human comprehension, and could be more concisely represented by numerical coordinates if readability was not an issue. A similar issue arrises when considering the numerical Internet Protocol addresses (IP addresses) used to identify computers and resources connected to the internet: IP addresses are convenient for use in networking protocols, but are difficult for humans to memorize. The Domain Name System (DNS) is in place to allow individuals or companies to register readable domain names (www.google.com) that can be mapped to the IP addresses needed to locate certain computers on the network, and is fundamental to the functionality of the internet.

The specific DNS protocol we examine in this report is referred to as a client lookup, as this protocol is vulnerable to an attack known as cache poisoning. A client lookup is initiated whenever an application on a client's computer sends a request that requires a domain name to be translated to an IP address. The application will first send a request to a domain name resolver, which will check its cache of recent lookups for a response to the request. If the cache contains an appropriate response, the resolver will return it to the sender of the request; if not, the resolver will send the request to the root name server. The root name server will check its cache, and either return a cached response, or tell the DNS resolver the location of the next name server it should query. This process is repeated until the resolver receives a final response, which it will cache and return to the client.

The ultimate goal of a cache poisoning attack is to trick the domain name resolver into caching an incorrect IP address, creating a false association between a domain name, like "www.google.com," and the location of the domain's servers. This in turn can divert users who try to visit the domain name in question to a malicious website. The process by which this attack is carried out begins with an attacker transmitting queries to the domain name resolver. The attacker then sends responses to the resolver that ordinarily come from name servers, but these responses have IP addresses that do not correspond to the domain name requested. The resolver will accept and cache these responses. Anyone who attempts to visit domains with compromised IP's before the cached responses expire will be directed to a computer of the attacker's choice. 


%------------------------------------------------

\section*{Implementation}

\subsection*{Node}

We realized quite early in the implementation that weren't going to be able to use methods with returns statements as a way to communication between servers. This is because there's no real way to intercept a return statement. The assignment of a variable to a non-void function will always result in whatever that particular function returns. The entirety of the simulation is based on forging messages (from an attacker to a DNS server), and as such we used the idea of calling another object's method with certain parameters as a way of sending information. 

This is why the attacker, the client, the DNS server and the name servers all implement the Node interface, which contains:

\begin{lstlisting}[language=Java]
public void message(Node src, Message message);
public String getAddress();
\end{lstlisting}

$getAddress$ is a helper function just used for checking the source of a response, which will be explained below. The message function, is utilized as follows. If $server_1$ wants to talk to $server_2$ and send them a message msg, the following code would be called by $server_1$:

\begin{lstlisting}[language=Java]
server_2.message(this, msg);	
\end{lstlisting}

Whatever the destination does with the message depends on the implementation, but this basically enables a race. In the case of the attacker and the name server, whomever calls the DNS' message function first and gets its answer accepted successfully caches that answer, be it malicious or not.

\subsection*{Message}

The message class encapsulates both a request and response into one object. What it acts as depends on the context and its construction. Its basic usage is to be able to send information across all of the nodes (the attacker, client and all the servers). There are multiple constructors for different purposes but the general class variables are these:

\begin{lstlisting}
private String type;
private Url query;
private String TXID;
private String answer;
private NameServer nextServer;	
\end{lstlisting}

The type is especially important as it determines the purpose of the message. A message of type $WHERE$ is recognized by the DNS server as a request from a client. A $TRY$ message is also recognized by the DNS as a message which contains the next relevant name server to query about the given URL. A $FINAL$ message is received by the DNS from a name server and then passed onto the client; this message type contains the IP address of the URL.

The query contains the URL as well as other components. The TXID is used as a method of checking the validity of the message, which is explained later on. The answer is the IP address (for $FINAL$ types) and $nextServer$ is used to store the next name server the DNS should query (for $TRY$ types).

\subsection*{Client}

The client is an important piece in the implementation, since it's essentially the target of the attack. The main objective is to have the DNS return a malicious IP from its cache when it receives a request from the Client. As such, the Client class implements the Node interface to be able to communicate with the DNS server through the message() function calls. 

The following is the class variables and the constructor for a Client object.

\begin{lstlisting}
private static final String TAG = "Client";	
private DNS dns_;
private Cache cache_;
    	
public Client(DNS dns) {
	this.dns_ = dns;
	this.cache_ = new Cache();
}	
\end{lstlisting}

The tag is included strictly for logging purposes. The constructor takes in a DNS server, as it should know, at the moment of its instantiation, what server it's going to refer to for requests. The cache is initialized right away and similarly to the DNS server, functions to avoid requesting a web address' IP it has recently queried. 

The following is the method we use to send a message to the DNS.

\begin{lstlisting}
public void makeRequest(Url query){
	// If not in cache, send request and wait for reply
	Message request = new Message(query);
	dns_.message(this, request);
}
\end{lstlisting}

This takes in a Url object, which contains the web address, encapsulates it in a Message object as a request, and calls the DNS' receive method, passing the request as a parameter. Note that the message method and the Message class are two different components of the implementation. One is used to package the information, the other to send/receive it. 

This leads me to the Client's implementation of the message method, following is an excerpt:

\begin{lstlisting}
// If we already have an entry in our cache, then we ignore any received messages
if (!cache_.isEmpty()){
    return; 
}
// Check that this is indeed a final answer from the DNS
if (message.getType() == MessageTypes.FINAL){
	// Add entry to cache
    Log.i(TAG, "Adding the following to cache: " + "{" + message.getQuery().toString() + " : " + message.getAnswer() + "}");
    cache_.addEntry(message.getQuery(), message.getAnswer());
}
else {
    return;
}
\end{lstlisting}

Once the DNS server has an answer for us, it’ll call this method. The message type will be FINAL since it contains the IP address that the client desires. Any other message types such as TRY and WHERE, which are meant only for the DNS, will be ignored. Once that message is received, we add the entry to our cache and log the event. 

All of these pieces are used together in visitWebPage, which returns the corresponding IP of a URL.

\begin{lstlisting}
public String visitWebPage(Url query){
    String address = cache_.lookupEntry(query);
    while (address == null){
    	makeRequest(query);
    	address = cache_.lookupEntry(query);
    }
    return address;
}	
\end{lstlisting}

Here we consistently check the cache to see if we have an entry for the corresponding query, making a new request every time we don’t. Eventually, once the DNS gets an answer back from the name servers or looks it up in its own cache, it’ll call the Client’s message function, which will put the answer in the cache and cause the function above to break out of the while loop and return the address. 

\subsection*{DNS}

The DNS is the main tool for the attack and the server whose cache gets poisoned. It operates in many ways similarly to the Client class, except that its message function is more sophisticated because of the interactions with the name servers.

These are its class variables:

\begin{lstlisting}
private static final String TAG = "DNS";	
private NameServer root_;
private Cache cache_;
private Node client_;
private Map<String, Integer> expectedServers_;
private Set<String> TXIDS_;
private final ReentrantLock lock_;
\end{lstlisting}

The tag is for logging purposes, which allows us to track the progress of the simulation. The next three variables, a root name server, a cache and a client are all essentials for the basic functionality. We require a root name server to send requests to, who respond with either the IP address or a reference to the another name server to ask. The cache is required to be poisoned with a malicious address that is sent to the client. The client instance is required to send this malicious address back. Of these three, only the root name server is required as a parameter for initialization, since the DNS must know who to initially ask about a URL. The client could be anyone trying to communicate with the server, but we’ll see how this is assigned later. 

The following two variables are used to ensure the legitimacy of name server responses. expectedServers is a hashmap which is used to keep track of the name server from whom we are expecting a response following a request. The string is the address of such server and the integer is used to keep count of how many responses we expect from that server. This is done to avoid an attacker to flood the DNS with responses that have not been requested. While this is used to verify the source of the response, TXIDS is used to verify the validity of the response itself. This set contains TXIDs which are created at the moment the DNS receives a request from a client and included in the message itself. Once the final response arrives from a name server with the address, the TXID in the message is checked against the set, and only accepted if it exists. What this means is that the attacker only has to be able to guess one TXID within the set to have its response be cached. Additionally, it will have to spoof the address, in order to pretend like it's coming from a legitimate name server, which we assume the attacker knows. Finally, the ReentrantLock is used to protect the TXID set from one thread attempting to traverse it while another is editing it. This lock uses queue based priority to ensure thread safety. 

Following is the excerpt within message that handles a client's request: 

\begin{lstlisting}
if (message.getType()==MessageTypes.WHERE){
    Log.i(TAG, "Received a message of type: " + message.getType());
    String TXID = Txid.genTxid();
    lock_.lock();
    while (TXIDS_.contains(TXID)){
    	TXID = Txid.genTxid();
    }
    TXIDS_.add(TXID);
    lock_.unlock();
    message.setTXID(TXID);
    this.client_ = src;            
// Now expecting a message back from root
    addToExpected(root_.getAddress());
    root_.message(this, message);           
}	
\end{lstlisting}

As can be seen, we also ensure that every single client request gets assigned a unique TXID by avoiding creating one that’s already in the set - all while maintaining thread safety. Since we know it was a client request due to the message type (WHERE), we assign the class variable of client to the source, effectively granting us the ability to contact it back. 

Lastly, we send a request to the root name server, to which we simply pass on the message given by the client, with the only modification being the addition of the TXID. The root server is also added to the expected server hashmap, as to accept its response. 

The next message type, TRY, is the name server’s response which refers to another name server to query. Once a TRY message goes through the two security checks (TXIDS and expectedServers), this occurs: 

\begin{lstlisting}
NameServer nextServer = message.getNextServer();
// Got message back from expected server, add next to the map 
decrementExpected(src.getAddress());
addToExpected(nextServer.getAddress());
nextServer.message(this, new Message(message.getQuery(), message.getTXID()));
\end{lstlisting}

The response contains the next server, which is extracted. The address of the server we just received from is subtracted from the expected count, and the next name server’s address is added to the expected. Finally, a new request is constructed and sent to the next name server. 

Once we get a FINAL message, the DNS has completed its objective: 

\begin{lstlisting}
decrementExpected(src.getAddress());
this.cache_.addEntry(message.getQuery(), message.getAnswer());
this.client_.message(this, message);
lock_.lock();
TXIDS_.remove(message.getTXID());
lock_.unlock();	
\end{lstlisting}

We decrement the count of expected messages from that particular server and cache the response. The final answer is sent to the client immediately after the caching and the TXID is removed from the set, as it has served its purpose.

So what happens if the client request an url that is already in the cache?

\begin{lstlisting}
if (message.getType() == MessageTypes.WHERE) {
    Url query = message.getQuery();
    Message toSend = new Message(query, cache_.lookupEntry(query), message.getTXID());
    src.message(this, toSend);
} else {
    return;
}
\end{lstlisting}

The first check is to make sure that this is coming from  client, and if it’s not, simply return. If it is, obtain the query and extract the address from the cache before packaging it into a response object (Message) and sending it back.

\subsection*{TTL}

The TTL (Time To Live) concept is essential when considering caches. A cache will only hold the information in it as long as the time to live allows it to. Moreover, a DNS server will ignore any responses from the name servers as long as its cache is occupied. Thus the attacker can only perform the exploit once every time the TTL expires. We created a class that holds these class variables. 

\begin{lstlisting}
private long expirationTime;
public static final long DEFAULT_TTL = 10000;
\end{lstlisting}

The rest is composed of an empty constructor that initializes the expirationTime to the default, and one that takes in an explicit value. Lastly, there’s a getter method for expirationTime. 

\subsection*{Cache}

The cache is an simple yet very important component of this project, which is why we decided to encapsulate it in its own class as opposed to just having a hashmap. Besides containing a map of Url to String, it also contains a map of Url to TTL:

\begin{lstlisting}
private Map<Url, String> cache;
private Map<TTL, Url> ttls;
\end{lstlisting}

We implemented methods to add to the cache when given a URL, IP address and a TTL, as well as check for containment. The most important component is the nested class, CachePurger, which extends the Java Runnable interface. The Java documentation describes this interface: “The Runnable interface should be implemented by any class whose instances are intended to be executed by a thread. The class must define a method of no arguments called run.” Our implemented class periodically visits the entries of a given cache and clears any cache entries whose TTLs have expired. We chose to start running this thread inside the constructor. The run method was implemented as follows:

\begin{lstlisting}
public void run() {
    while (true) {
	    sleepUntilUnset(entering);
        for (TTL ttl : ttls.keySet()) {
            if (getCurrentTime() >= ttl.getExpTime()) {
            lock.lock()
            Url expired = ttls.get(ttl);
            remove(expired);
            ttls.remove(ttl);
            lock.unlock()
            }
        }
    // Only do this once per second
    try {
        Thread.sleep(1000);
    } catch (InterruptedException e) {
        run();
    }
    }
}
\end{lstlisting}

The TTL map is used to iterate across all the entries and compare with the current time. Once one is found that has expired, the corresponding Url is retrieved out of the TTL map and removed from the cache. The thread then sleep for a second before performing the check again. The rest of the methods are getters and setters. 

\subsection*{TXID}

TXIDs are used to verify the validity of a package, and in line with the the standard DNS implementations, it’s a 16 bit digit. This class contains a static method that just returns a string of 16 random bits.  

\subsection*{Log}

This class contains static logging methods that provide ease and flexibility when tracking the activity of the simulation. There are three logging methods, all of which take a TAG and an a String. The tag is used to identify where the logging message is coming from (DNS, Client, Name Server or Attacker).

This uses Java's System.err.println() to print out a warning:

\begin{lstlisting}
public static void w(String tag, String warning)
\end{lstlisting}

\begin{lstlisting}
public static void e(String tag, String error) 
\end{lstlisting}

Is similar to w except it it calls System.exit(1) after logging, effectively shutting down the simulation, due to an error. 
\begin{lstlisting}
public static void i(String tag, String info) 
\end{lstlisting}

This simply prints out certain information to keep the user updated with the progress. For example, it’s used to notify when the DNS server has received a request from a client. 


\subsection*{Name Server}

The NameServer class is initialized by each Simulation. The simulations pass it a list of known web pages and a list of trusted DNS servers. The first step is to preprocess each URL wntry, which consists of a raw address and IP address. We created a Url class to allow us to easily store and manipulate the parts of a URL. An instance of the URL class stores an array of "parts", that is to say a segmented version of the raw address by the dots. We have some methods to help the name server find the next name server to tell the DNS server to talk to, which we will get to once we describe the respective methods in the NameServer class.

The first step is to build a tree out of the list of URLs. We represent each node of the tree as a Map from Strings to NameServers. We apply the following method for each URL (to the root NameServer).

\begin{lstlisting}
private void addEntry(Url url) {
    if (url.isDepleted()) {
        this.lookup = url.getAddress();
        return;
    }
    String segment = url.nextSegment();
    url.incrementPointer();
    if (children.containsKey(segment)) {
        NameServer child = children.get(segment);
        child.addEntry(url);
    } else {
        NameServer child = new NameServer(...);
        children.put(segment, child);
        child.addEntry(url);
    }
}
\end{lstlisting}

The URL class points to the current "part" of the raw address we are on. If we've reached the end of a raw address, then we set the address entry of the current NameServer to the IP address specified by the URL instance. Otherwise, we query the URL class for the next portion of the address. If the children map already contains the key, we iterate through the URL, get the child NameServer from the map, and recursively apply this method to the child. If the key is not present, then we initialize a new NameServer using a private constructor. This private constructor takes in a raw address segment (so that we can keep track of which raw address this NameServer is meant to resolve queries for), along with the relevant instance variables of the parent NameServer. Once the child NameServer is initialized and added to the map, we recursively invoke the method on what remains of the URL.

We now describe what occurs when a DNS server queries the root name server. Note that this is an iterative implementation of DNS (as opposed to a recursive implementation). That is to say the NameServer tells the DNS server which NameServer to query next if it does not have the answer. The first step is to verify that the DNS server talking to us is a trusted server. We also make sure that the query is in the right format. Next, we extract the queried raw address from the message. In the case that we are the name server that knows the IP address of this raw URL, we return a "final" message as a response with the IP address. Otherwise, we verify that this NameServer is on the right path through the tree from the root for the URL. If we are, then we use the URL class to find the next segment of the URL. We query our children map for the NameServer corresponding to the next segment and return its address as a response to the DNS server. We are careful to appropriately handle edge cases, such as when the child is not found or we are not in the path of the raw address.

\subsection*{The Attacker}

The attacker holds four fields: a target DNS server, the root NameServer, a malicious URL, and a legitimite URL to target. The constructor, invoked by the attacker simulation, takes the fields as input and sets them. The most important method is the attack method.

\begin{lstlisting}
public void attack(Url fake, Url legitimite) {
    this.malicious = fake;
    this.legitimite = legitimite;
    
    // send requests and spoofed responses until successful
    while (!successful) {
        new Thread(new Request(this)).start();
        new Thread(new Response(this)).start();
    }
}
\end{lstlisting}

The attack method takes a desired malicious URL and legitimate URL to target and begins the flooding. When the attacker receives a response from the DNS server to one of its requests indicating that it has cached the malicious URL, the attacker sets its "successful" field to true to stop creating new threads and stops all currently running threads.

Notice that the code above creates two threads each with a different runnable: Request and Response. The nested Request class sends a query to the DNS server for the legitimate address. The nested Response class creates a fake response by spoofing the authoritative name server's address and randomly generating a TXID. It states that the address of the malicious URL is the address of the fake URL. Thanks to the birthday paradox, there is a decent probability of getting a match after a couple hundred spoofed requests and responses.

\subsection*{Unfettered Simulation}

The first goal of the project was to get a normal simulation up and running, with no attacker or poisoning involved. This is precisely what this class is intended for. It has the potential to be poisoned, but that is done only in the AttackerSimulation. The class variables are these:

\begin{lstlisting}
private List<Url> pagesToVisit;
private String knownWebPages;
private Client client;
private NameServer root;
private DNS dns;
\end{lstlisting}

The list of Urls is used to keep all the URLs that we want our simulation to "visit", that is, retrieve their respective IP addresses. The following String, knownWebPages, is just the name of the file that contains a list of all web pages in our simulation of the internet. It's used for the initialization of the name servers, refer to that section for a more detailed explanation. The rest are quite self-explanatory.

The constructor takes in a list of web pages to visit and the knownWebPages file name for the name server constructor. This is occurs inside:
\begin{lstlisting}
this.pagesToVisit = pagesToVisit;
this.knownWebPages = knownWebPages;
DNS dns = new DNS();
this.dns = dns;
Set<Node> dnsServers = new HashSet<Node>();
dnsServers.add(dns);
NameServer rootNS = new NameServer(knownWebPages, dnsServers, "128.532.543.645");
this.root = rootNS;
dns.init(rootNS);
this.client = new Client(dns);
\end{lstlisting}

It begins creating the DNS server and adding it to a set, which is then passed into the constructor for the root name server. This was done via a set because future work could include implementing the simulation with multiple DNS servers, which would be easy to extend. The root server is initialized with the knownWebPages, DNS servers and a given address, which we hardcoded. Since the root server requires DNS servers to be initialized, but the DNS server itself requires a root server for its own creation, we delay the task to DNS’ init() method, instead of the constructor. Lastly, we create the client by passing in the DNS server it will refer to.

The method unfettered(), runs the simulation by iterating through the list of web pages and calls the client’s visitWebPage on them. 

\begin{lstlisting}
for (Url url : pagesToVisit) {
	String answer = client.visitWebPage(url);
	System.out.println("\tClient received IP successfully!");
}
\end{lstlisting}

The main method puts all the pieces together to execute the simulation:

\begin{lstlisting}
public static void main(String[] args) {
	System.out.println("---------- Unfettered Simulation ----------");
	List<Url> pagesToVisit = new LinkedList<Url>();
	pagesToVisit.add(new Url("www.oberlin.edu", "192.168.1.1"));
	new Simulation(pagesToVisit, "addresses.txt").unfettered();
		System.out.println("---------- Unfettered Simulation End -------");
}
\end{lstlisting}




\subsection*{Attack Simulation}

The attacker simulation first initializes a set of web pages to target using the URL class. We then initialize the necessary components to start an unfettered simulation, but, before starting it, we bring in an attacker. We initialize the attacker and invoke the attack method with a malicious URL passed as the target along with the web pages the unfettered simulation plans on visiting. Once the simulation determines that the attacker successfully poisoning the cache, we begin the unfettered simulation. If all went well, the simulation will go a bit differently this time. The returned address to the client's query should be the malicious address.

After pages of spam by the DNS server and Name Servers frantically handling the influx of requests, we see the following in our logs. (111.111.111.111 is our placeholder malicious address)

\begin{verbatim}
    DNS -- Adding the following to cache: {www.malicious.com:111.111.111.111}
\end{verbatim}

This indicates that the attacker can now stop. Of course, all the requests in progress are still jumping between the NameServers and DNS servers. They take a bit of time to subside, and the output sometimes continues happening even as the unfettered simulation is running. This can lead to some confusion logs, and a future goal could be to clean them up.

%------------------------------------------------

\section*{Conclusion}

Fusce in nibh augue. Cum sociis natoque penatibus et magnis dis parturient montes, nascetur ridiculus mus. In dictum accumsan sapien, ut hendrerit nisi. Phasellus ut nulla mauris. Phasellus sagittis nec odio sed posuere. Vestibulum porttitor dolor quis suscipit bibendum. Mauris risus lectus, cursus vitae hendrerit posuere, congue ac est. Suspendisse commodo eu eros non cursus. Mauris ultrices venenatis dolor, sed aliquet odio tempor pellentesque. Duis ultricies, mauris id lobortis vulputate, tellus turpis eleifend elit, in gravida leo tortor ultricies est. Maecenas vitae ipsum at dui sodales condimentum a quis dui. Nam mi sapien, lobortis ac blandit eget, dignissim quis nunc.

\begin{enumerate}
\item First numbered list item
\item Second numbered list item
\end{enumerate}



%----------------------------------------------------------------------------------------
%	BIBLIOGRAPHY
%----------------------------------------------------------------------------------------
%
%\bibliographystyle{unsrt}
%
%\bibliography{sample}

%----------------------------------------------------------------------------------------

\end{document}
