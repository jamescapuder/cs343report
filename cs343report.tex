%%%%%%%%%%%%%%%%%%%%%%%%%%%%%%%%%%%%%%%%%
% Thin Sectioned Essay
% LaTeX Template
% Version 1.0 (3/8/13)
%
% This template has been downloaded from:
% http://www.LaTeXTemplates.com
%
% Original Author:
% Nicolas Diaz (nsdiaz@uc.cl) with extensive modifications by:
% Vel (vel@latextemplates.com)
%
% License:
% CC BY-NC-SA 3.0 (http://creativecommons.org/licenses/by-nc-sa/3.0/)
%
%%%%%%%%%%%%%%%%%%%%%%%%%%%%%%%%%%%%%%%%%

%----------------------------------------------------------------------------------------
%	PACKAGES AND OTHER DOCUMENT CONFIGURATIONS
%----------------------------------------------------------------------------------------

\documentclass[a4paper, 12pt]{article} % Font size (can be 10pt, 11pt or 12pt) and paper size (remove a4paper for US letter paper)

\usepackage[protrusion=true,expansion=true]{microtype} % Better typography
\usepackage{graphicx} % Required for including pictures
\usepackage{wrapfig} % Allows in-line images

\usepackage{mathpazo} % Use the Palatino font
\usepackage[T1]{fontenc} % Required for accented characters
\linespread{1.25} % Change line spacing here, Palatino benefits from a slight increase by default
%\usepackage[1.0in]{geometry}
\makeatletter
\renewcommand\@biblabel[1]{\textbf{#1.}} % Change the square brackets for each bibliography item from '[1]' to '1.'
\renewcommand{\@listI}{\itemsep=0pt} % Reduce the space between items in the itemize and enumerate environments and the bibliography

\renewcommand{\maketitle}{ % Customize the title - do not edit title and author name here, see the TITLE block below
\begin{flushright} % Right align
{\LARGE\@title} % Increase the font size of the title

\vspace{50pt} % Some vertical space between the title and author name

{\large\@author} % Author name
\\\@date % Date

\vspace{40pt} % Some vertical space between the author block and abstract
\end{flushright}
}

%----------------------------------------------------------------------------------------
%	TITLE
%----------------------------------------------------------------------------------------

\title{\textbf{Exploring DNS Security}\\ % Title
Focused and Deliciously Witty Subtitle} % Subtitle

\author{\textsc{J. Berretta, J. Capuder, E. Guendel} % Author
\\{\textit{Oberlin College}}} % Institution

\date{\today} % Date

%----------------------------------------------------------------------------------------

\begin{document}

\maketitle % Print the title section

%----------------------------------------------------------------------------------------
%	ABSTRACT AND KEYWORDS
%----------------------------------------------------------------------------------------

%\renewcommand{\abstractname}{Summary} % Uncomment to change the name of the abstract to something else

\begin{abstract}
The security of common internet protocols is an important issue when considering how to satisfy the requirements of security and privacy as technology becomes more integral to our daily lives. The components required to simulate a DNS lookup were implemented in Java. This simulation was used to attempt a common DNS cache poisoning attack, and explore methods for improving the security of DNS protocol.
\end{abstract}

%\hspace*{3,6mm}\textit{Keywords:} lorem , ipsum , dolor , sit amet , lectus % Keywords

\vspace{10pt} % Some vertical space between the abstract and first section

%----------------------------------------------------------------------------------------
%	ESSAY BODY
%----------------------------------------------------------------------------------------

\section*{Introduction}

One of the most important requirements of any communication network, such as the internet or the United States Postal Service, is the ability to uniquely identify and locate participants of the network. This requirement is often met by assigning some form of address to participants: to determine the location to which an outgoing letter should be delivered, the sender must specify country of the recipient, what part of the country they live in, and finally the specific street/number combination at which the recipient resides. This system, however, was designed for human comprehension, and could be more concisely represented by numerical coordinates if readability was not an issue. A similar issue arrises when considering the numerical Internet Protocol addresses (IP addresses) used to identify computers and resources connected to the internet: IP addresses are convenient for use in networking protocols, but are difficult for humans to memorize. The Domain Name System (DNS) is in place to allow individuals or companies to register readable domain names (www.google.com) that can be mapped to the IP addresses needed to locate certain computers on the network, and is fundamental to the functionality of the internet.

When considering the secuirty of systems as important to the internet as the DNS, one might hope that the conventional principles of \textbf{C.I.A.}\footnote{\textbf{Confidentiality:} Only authorized parties should be able to view data; \textbf{Integrity:} Data should not be altered in any unauthorized fashion; \textbf{Availability:} Data should be available to and modifiable by authorized individuals.}  





%------------------------------------------------

\section*{Section Name}

Cras gravida, est vel interdum euismod, tortor mi lobortis mi, quis adipiscing elit lacus ut orci. Phasellus nec fringilla nisi, ut vestibulum neque. Aenean non risus eu nunc accumsan condimentum at sed ipsum.
%\begin{wrapfigure}{l}{0.4\textwidth} % Inline image example
%\begin{center}
%\includegraphics[width=0.38\textwidth]{fish.png}
%\end{center}
%\caption{Fish}
%\end{wrapfigure}
Aliquam fringilla non diam sed varius. Suspendisse tellus felis, hendrerit non bibendum ut, adipiscing vitae diam. Lorem ipsum dolor sit amet, consectetur adipiscing elit. Nulla lobortis purus eget nisl scelerisque, commodo rhoncus lacus porta. Vestibulum vitae turpis tincidunt, varius dolor in, dictum lectus. Aenean ac ornare augue, ac facilisis purus. Sed leo lorem, molestie sit amet fermentum id, suscipit ut sem. Vestibulum orci arcu, vehicula sed tortor id, ornare dapibus lorem. Praesent aliquet iaculis lacus nec fermentum. Morbi eleifend blandit dolor, pharetra hendrerit neque ornare vel. Nulla ornare, nisl eget imperdiet ornare, libero enim interdum mi, ut lobortis quam velit bibendum nibh.

Morbi tempor congue porta. Proin semper, leo vitae faucibus dictum, metus mauris lacinia lorem, ac congue leo felis eu turpis. Sed nec nunc pellentesque, gravida eros at, porttitor ipsum. Praesent consequat urna a lacus lobortis ultrices eget ac metus. In tempus hendrerit rhoncus. Mauris dignissim turpis id sollicitudin lacinia. Praesent libero tellus, fringilla nec ullamcorper at, ultrices id nulla. Phasellus placerat a tellus a malesuada.

%------------------------------------------------

\section*{Conclusion}

Fusce in nibh augue. Cum sociis natoque penatibus et magnis dis parturient montes, nascetur ridiculus mus. In dictum accumsan sapien, ut hendrerit nisi. Phasellus ut nulla mauris. Phasellus sagittis nec odio sed posuere. Vestibulum porttitor dolor quis suscipit bibendum. Mauris risus lectus, cursus vitae hendrerit posuere, congue ac est. Suspendisse commodo eu eros non cursus. Mauris ultrices venenatis dolor, sed aliquet odio tempor pellentesque. Duis ultricies, mauris id lobortis vulputate, tellus turpis eleifend elit, in gravida leo tortor ultricies est. Maecenas vitae ipsum at dui sodales condimentum a quis dui. Nam mi sapien, lobortis ac blandit eget, dignissim quis nunc.

\begin{enumerate}
\item First numbered list item
\item Second numbered list item
\end{enumerate}



%----------------------------------------------------------------------------------------
%	BIBLIOGRAPHY
%----------------------------------------------------------------------------------------
%
%\bibliographystyle{unsrt}
%
%\bibliography{sample}

%----------------------------------------------------------------------------------------

\end{document}
