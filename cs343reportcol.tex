%%%%%%%%%%%%%%%%%%%%%%%%%%%%%%%%%%%%%%%%%
% Journal Article
% LaTeX Template
% Version 1.4 (15/5/16)
%
% This template has been downloaded from:
% http://www.LaTeXTemplates.com
%
% Original author:
% Frits Wenneker (http://www.howtotex.com) with extensive modifications by
% Vel (vel@LaTeXTemplates.com)
%
% License:
% CC BY-NC-SA 3.0 (http://creativecommons.org/licenses/by-nc-sa/3.0/)
%
%%%%%%%%%%%%%%%%%%%%%%%%%%%%%%%%%%%%%%%%%

%----------------------------------------------------------------------------------------
%	PACKAGES AND OTHER DOCUMENT CONFIGURATIONS
%----------------------------------------------------------------------------------------

\documentclass[twoside,twocolumn]{article}

\usepackage{blindtext} % Package to generate dummy text throughout this template 

\usepackage[sc]{mathpazo} % Use the Palatino font
\usepackage[T1]{fontenc} % Use 8-bit encoding that has 256 glyphs
\linespread{1.05} % Line spacing - Palatino needs more space between lines
\usepackage{microtype} % Slightly tweak font spacing for aesthetics

\usepackage[english]{babel} % Language hyphenation and typographical rules

\usepackage[hmarginratio=1:1,top=32mm,columnsep=20pt]{geometry} % Document margins
\usepackage[hang, small,labelfont=bf,up,textfont=it,up]{caption} % Custom captions under/above floats in tables or figures
\usepackage{booktabs} % Horizontal rules in tables

\usepackage{lettrine} % The lettrine is the first enlarged letter at the beginning of the text

\usepackage{enumitem} % Customized lists
\setlist[itemize]{noitemsep} % Make itemize lists more compact

\usepackage{abstract} % Allows abstract customization
\renewcommand{\abstractnamefont}{\normalfont\bfseries} % Set the "Abstract" text to bold
\renewcommand{\abstracttextfont}{\normalfont\small\itshape} % Set the abstract itself to small italic text

\usepackage{titlesec} % Allows customization of titles
\renewcommand\thesection{\Roman{section}} % Roman numerals for the sections
\renewcommand\thesubsection{\roman{subsection}} % roman numerals for subsections
\titleformat{\section}[block]{\large\scshape\centering}{\thesection.}{1em}{} % Change the look of the section titles
\titleformat{\subsection}[block]{\large}{\thesubsection.}{1em}{} % Change the look of the section titles

\usepackage{fancyhdr} % Headers and footers
\pagestyle{fancy} % All pages have headers and footers
\fancyhead{} % Blank out the default header
\fancyfoot{} % Blank out the default footer
\fancyhead[C]{Running title $\bullet$ May 2016 $\bullet$ Vol. XXI, No. 1} % Custom header text
\fancyfoot[RO,LE]{\thepage} % Custom footer text

\usepackage{titling} % Customizing the title section

\usepackage{hyperref} % For hyperlinks in the PDF

%----------------------------------------------------------------------------------------
%	TITLE SECTION
%----------------------------------------------------------------------------------------

\setlength{\droptitle}{-4\baselineskip} % Move the title up

\pretitle{\begin{center}\Huge\bfseries} % Article title formatting
\posttitle{\end{center}} % Article title closing formatting
\title{Exploring DNS Security} % Article title
\author{%
\textsc{James Capuder} \\[1ex] % Your name
\normalsize Oberlin College \\ % Your institution
%\normalsize \href{mailto:john@smith.com}{john@smith.com} % Your email address
\and % Uncomment if 2 authors are required, duplicate these 4 lines if more
\textsc{Eren Guendel} \\[1ex] % Second author's name
\normalsize Oberlin College \\ % Second author's institution
%\normalsize \href{mailto:jane@smith.com}{jane@smith.com} % Second author's email address
\and % Uncomment if 2 authors are required, duplicate these 4 lines if more
\textsc{Juanbi Berretta} \\[1ex] % Second author's name
\normalsize Oberlin College \\ % Second author's institution
%\normalsize \href{mailto:jane@smith.com}{jane@smith.com} % Second author's email address
}
\date{\today} % Leave empty to omit a date
\renewcommand{\maketitlehookd}{%
\begin{abstract}
\noindent The security of common internet protocols is an important issue when considering how to satisfy the requirements of security and privacy as technology becomes more integral to our daily lives. The components required to simulate a DNS lookup were implemented in Java. This simulation was used to attempt a common DNS cache poisoning attack, and explore methods for improving the security of DNS protocol.
 % Dummy abstract text - replace \blindtext with your abstract text
\end{abstract}
}

%----------------------------------------------------------------------------------------

\begin{document}

% Print the title
\maketitle

%----------------------------------------------------------------------------------------
%	ARTICLE CONTENTS
%----------------------------------------------------------------------------------------

\section{Introduction}

\lettrine[nindent=0em,lines=3]{O} ne of the most important requirements of any communication network, such as the internet or the United States Postal Service, is the ability to uniquely identify and locate participants of the network. This requirement is often met by assigning some form of address to participants: to determine the location to which an outgoing letter should be delivered, the sender must specify country of the recipient, what part of the country they live in, and finally the specific street/number combination at which the recipient resides. This system, however, was designed for human comprehension, and could be more concisely represented by numerical coordinates if readability was not an issue. A similar issue arrises when considering the numerical Internet Protocol addresses (IP addresses) used to identify computers and resources connected to the internet: IP addresses are convenient for use in networking protocols, but are difficult for humans to memorize. The Domain Name System (DNS) is in place to allow individuals or companies to register readable domain names (www.google.com) that can be mapped to the IP addresses needed to locate certain computers on the network, and is fundamental to the functionality of the internet.

The specific DNS protocol we examine in this report is referred to as a client lookup, as this protocol is vulnerable to an attack known as cache poisoning. A client lookup is initiated whenever an application on a client's computer sends a request that requires a domain name to be translated to an IP address. The application will first send a request to a domain name resolver, which will check its cache of recent lookups for a response to the request. If the cache contains an appropriate response, the resolver will return it to the sender of the request; if not, the resolver will send the request to the root name server. The root name server will check its cache, and either return a cached response, or tell the DNS resolver the location of the next name server it should query. This process is repeated until the resolver receives a final response, which it will cache and return to the client.

The ultimate goal of a cache poisoning attack is to trick the domain name resolver into caching an incorrect IP address, creating a false association between a domain name, like "www.google.com," and the location of the domain's servers. This in turn can divert users who try to visit the domain name in question to a malicious website. The process by which this attack is carried out begins with an attacker transmitting queries to the domain name resolver. The attacker then sends responses to the resolver that ordinarily come from name servers, but these responses have IP addresses that do not correspond to the domain name requested. The resolver will accept and cache these responses. Anyone who attempts to visit domains with compromised IP's before the cached responses expire will be directed to a computer of the attacker's choice. 

%------------------------------------------------

\section{Implementation}

Maecenas sed ultricies felis. Sed imperdiet dictum arcu a egestas. 
\begin{itemize}
\item Donec dolor arcu, rutrum id molestie in, viverra sed diam
\item Curabitur feugiat
\item turpis sed auctor facilisis
\item arcu eros accumsan lorem, at posuere mi diam sit amet tortor
\item Fusce fermentum, mi sit amet euismod rutrum
\item sem lorem molestie diam, iaculis aliquet sapien tortor non nisi
\item Pellentesque bibendum pretium aliquet
\end{itemize}
\blindtext % Dummy text

Text requiring further explanation\footnote{Example footnote}.

%------------------------------------------------

\section{Results}

\begin{table}
\caption{Example table}
\centering
\begin{tabular}{llr}
\toprule
\multicolumn{2}{c}{Name} \\
\cmidrule(r){1-2}
First name & Last Name & Grade \\
\midrule
John & Doe & $7.5$ \\
Richard & Miles & $2$ \\
\bottomrule
\end{tabular}
\end{table}

\blindtext % Dummy text

\begin{equation}
\label{eq:emc}
e = mc^2
\end{equation}

\blindtext % Dummy text

%------------------------------------------------

\section{Discussion}

\subsection{Subsection One}

A statement requiring citation \cite{Figueredo:2009dg}.
\blindtext % Dummy text

\subsection{Subsection Two}

\blindtext % Dummy text

%----------------------------------------------------------------------------------------
%	REFERENCE LIST
%----------------------------------------------------------------------------------------

\begin{thebibliography}{99} % Bibliography - this is intentionally simple in this template

\bibitem[Figueredo and Wolf, 2009]{Figueredo:2009dg}
Figueredo, A.~J. and Wolf, P. S.~A. (2009).
\newblock Assortative pairing and life history strategy - a cross-cultural
  study.
\newblock {\em Human Nature}, 20:317--330.
 
\end{thebibliography}

%----------------------------------------------------------------------------------------

\end{document}
